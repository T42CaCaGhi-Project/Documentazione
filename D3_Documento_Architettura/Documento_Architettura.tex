\documentclass{article}

\usepackage[utf8]{inputenc}
\usepackage{amsmath}
\usepackage{amsfonts}
\usepackage{amssymb}
\usepackage{graphicx}
\usepackage[table,xcdraw]{xcolor}
\usepackage[hidelinks]{hyperref}
\usepackage{fontawesome5}
\usepackage{longtable}


\graphicspath{ {./images/} }

\renewcommand{\contentsname}{Indice}

\makeatletter
\newcommand*{\rom}[1]{\expandafter\@slowromancap\romannumeral #1@}
\makeatother

\usepackage[a4paper,top=2cm,bottom=2cm,left=2cm,right=2cm]{geometry}


\title{\textbf{\Huge Documento di Architettura}}
\author{Edoardo Ghirardello, Giulio Cappelli, Elia Casotti \\ \\ Gruppo T42}
\date{2022}

\let\origthesubsection\thesubsection

\begin{document}

\maketitle

\clearpage
\tableofcontents
\clearpage

\section{Scopo del documento}
\begin{description}
    \item[] Nel presente documento viene definita l'architettura del sistema "Fen Festa" utilizzando il Class Diagram UML e codice in OCL (Object Constraint Language). \\
        Nel precedente documento sono stati rappresentati i componenti del sistema che ora verranno tradotti in classi attraverso il diagramma UML indicato.
\end{description}
\clearpage
\section{Diagramma delle Classi}
\begin{description}
    \item[]
\end{description}
\clearpage
\section{Codice in Object Constraint Language}
\begin{description}
    \item[]
\end{description}
\clearpage
\section{Diagramma delle Classi con codice OCL}
\begin{description}
    \item[]
\end{description}
\end{document}