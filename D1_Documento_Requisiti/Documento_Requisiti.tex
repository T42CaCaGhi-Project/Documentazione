\documentclass{article}

\usepackage[utf8]{inputenc}
\usepackage{amsmath}
\usepackage{amsfonts}
\usepackage{amssymb}
\usepackage{graphicx}
\usepackage[table,xcdraw]{xcolor}
\usepackage[hidelinks]{hyperref}


\graphicspath{ {./images/} }

\renewcommand{\contentsname}{Indice}

\makeatletter
\newcommand*{\rom}[1]{\expandafter\@slowromancap\romannumeral #1@}
\makeatother

\usepackage[a4paper,top=2cm,bottom=2cm,left=2cm,right=2cm]{geometry}

\setcounter{section}{-1}

\title{\textbf{\Huge Analisi dei Requisiti}}
\author{Edoardo Ghirardello, Giulio Cappelli, Elia Casotti \\ Gruppo T42}
\date{2022}


\begin{document}

\maketitle

\clearpage
\tableofcontents
\clearpage

\section{Scopo del documento}
\begin{description}
    \item[] In questo documento si riporta la descrizione e il funzionamento dell'applicazione "NOME" in linguaggio naturale.
    \item[] Verranno analizzati gli obiettivi del progetto, definiti i requisiti funzionali e non funzionali e infine verranno esposti i requisiti di front-end e di back-end.
\end{description}
\clearpage
\section{Obiettivi del progetto}
\begin{description}
    \item[] Data la complessità della ricerca degli eventi presenti a Trento, siamo giunti alla conclusione che sarebbe utile un'applicazione raggruppante tutti gli eventi della serata, presenti in città e dintorni.
    \item[] Nello specifico le funzionalità prevedono che:
        \begin{itemize}
            \item L'\textbf{utente semplice} possa visualizzare un calendario (Mensile, Settimanale, Giornaliero) nel quale verranno segnati gli eventi, visualizzati con Titolo e Immagine;
            \item L'utente, una volta selezionato un evento, possa visualizzarne la descrizione (inclusa la posizione visualizzata tramite una mappa).
            \item Effettuando l'accesso (\textbf{utente registrato}), l'utente possa:
                  \begin{itemize}
                      \item Salvarsi gli eventi (Preferiti);
                      \item Ricevere notifiche (via mail)
                      \item Salvarsi le proprie preferenze (Filtri);
                      \item Indicare se parteciperà o meno all'evento;
                      \item Visualizzare tutte le funzionalità dell'utente semplice;
                  \end{itemize}
            \item L'\textbf{utente con privilegi} possa inserire, modificare ed eliminare eventi. \\
                  L'evento sarà creato tramite un form precompilato:
                  \begin{itemize}
                      \item Immagine;
                      \item Descrizione;
                      \item Tag;
                      \item Orario;
                      \item Luogo (Posizione/Indirizzo);
                  \end{itemize}
            \item L'\textbf{utente amministratore} abbia privilegi unici quali:
                  \begin{itemize}
                      \item Consegna e revoca dei privilegi agli altri utenti;
                      \item Cancellazione di eventi;
                      \item Possibilità d'inviare avvisi di manutenzione, o simili, agli altri utenti loggati (mail);
                      \item Accesso a tutte le altre funzionalità;
                  \end{itemize}
        \end{itemize}
\end{description}
\clearpage
\end{document}